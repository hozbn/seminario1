\documentclass{IEEEtran}
\usepackage[spanish]{babel}
\usepackage[utf8]{inputenc}
\usepackage{authblk}
\usepackage{amsmath}
\usepackage{subfig}
\usepackage{graphicx}
\usepackage{cite}

\title{Investigación y Evaluación de Estrategias de Obtención de Mediciones EIT para Rastreo de Dispersión de líquidos. }
\author[1]{O. F. Cándido Sánchez}
\affil[1]{Departamento de Ingeniería Electrónica, Instituto Tecnológico de Morelia}


\begin{document}
  \maketitle
    %Abstract
  \begin{abstract}
  \end{abstract}

  %Keywords
  \begin{IEEEkeywords}
  \end{IEEEkeywords}

  \section{Introduccion}
En la última década, la tomografía de impedancia eléctrica (Electrical Impedance Tomography: \textit{EIT}) ha recibido considerable atención por parte de la comunidad científica en el mundo para aplicaciones médicas e industriales.En medicina, EIT es una herramienta que puede ser aplicable para rastrear la difusion de medicamentos quimioterapeúticos para tratamiento de cáncer de mama, de tal forma que se pueden proponer modelos matemáticos por aplicación locorregional \cite{Gnecchi2018}.\\
En general, el objetivo de la \textit{EIT} es el de reconstruir imágenes, las cuales representan una seccion transversal de una distribucion espacial de impedancia eléctrica interna de un objeto, ya sea en dos o tres dimensiones \cite{Gnecchi2012}.\\

El estudio de cancer de mama por EIT está aprobado por la FDA para ayudar a clasificar los tumores encontrados en los mamogramas. Sin embargo, hasta el momento no se han realizado suficientes pruebas clínicas para que se pueda usar en pruebas de detección del cáncer de seno \cite{Cancer.org}.La EIT podría usarse como complemento de la mamografía y la ecografía para la detección del cáncer de mama. Sin embargo, la diferenciación de las lesiones malignas de las benignas en función de las mediciones de impedancia requiere más investigación.\cite{Zou2003}.\\

Este trabajo está dedicado a proponer una metodología para obtener, reconstruir y analizar mediciones EIT que permitan proponer modelos matemáticos de procesos que involucren la dispersión de líquidos en medios permeables.\\


  \section{Theoretically section}
  \section{Methodology}
  \section{Results and discussion}
  \section{Conclusion}
\bibliographystyle{IEEEtran}
\bibliography{bibliography}

\end{document}
